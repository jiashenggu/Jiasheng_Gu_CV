% --- LaTeX CV Template - S. Venkatraman ---

% Set document class and font size
\documentclass[letterpaper, 11pt]{article}
% \documentclass[11pt, letterpaper]{awesome-cv}
% \usepackage[utf8]{inputenc}

% Package imports
\usepackage{setspace, longtable, graphicx, hyphenat, hyperref, fancyhdr, ifthen, enumitem, amsmath, setspace}

% --- Page layout settings ---

% Set page margins
\usepackage[left=1in, right=1in, bottom=0.7in, top=0.7in]{geometry}


% Defines writer's mobile (optional)
% Usage: \mobile{<mobile number>}
\newcommand*{\mobile}[1]{\def\@mobile{#1}}

% Defines writer's email (optional)
% Usage: \email{<email adress>}
\newcommand*{\email}[1]{\def\@email{#1}}

% Defines writer's homepage (optional)
% Usage: \homepage{<url>}
\newcommand*{\homepage}[1]{\def\@homepage{#1}}

% Defines writer's github (optional)
% Usage: \github{<github-nick>}
\newcommand*{\github}[1]{\def\@github{#1}}

% Set line spacing
\renewcommand{\baselinestretch}{1.15}

% --- Page formatting ---

% Set link colors
\usepackage[dvipsnames]{xcolor}
\hypersetup{colorlinks=true, linkcolor=RoyalBlue, urlcolor=RoyalBlue}

% Set font to Libertine, including math support
\usepackage{libertine}
\usepackage[libertine]{newtxmath}

% Remove page numbering
\pagenumbering{gobble}

% --- Document starts here ---

\begin{document}

% Name and date of last update to this document
\noindent{\Huge{Jiasheng Gu}
\hfill{\it\footnotesize Updated \today}}

% --- Contact information and other items ---

\vspace{0.5cm} 
\begin{center}
\begin{tabular}{lllll}
% Line 1: Email, GitHub, office location
\textbf{Email}: gujiashe@usc.edu      &
 \href{https://github.com/jiashenggu}{\textbf{GitHub}}    &
 \textbf{Phone}: +1 (213) 204-0294   & 
 \href{https://www.linkedin.com/in/jiasheng-gu/}{\textbf{LinkedIn}} &
 \href{https://jiashenggu.github.io/}{\textbf{Homepage}} 
\end{tabular}
\end{center}

% --- Start the two-column table storing the main content ---

% Set spacing between columns
\setlength{\tabcolsep}{8pt}

% Set the width of each column
\begin{longtable}{p{1.3in}p{4.8in}}


% --- Section: Research interests ---

\nohyphens{\color{black}{Research interests}}
& I have a broad interest in natural language processing, machine learning, and artificial intelligence, with a particular interest in text generation, trustworthy AI models, and few-shot NLP. \\
& \\

% --- Section: Education ---


% & \textbf{University 1} \hfill City, State \\ 
% & PhD in Subject \hfill Month Year -- Present \\
% & Mentors: Professors A, B. {\it GPA: X.YZ.}\\
% & \\

\color{black}{Education} & \textbf{University of Southern California} \hfill Los Angeles, CA \\
& M.S. in Machine Learning and Data Science \hfill Aug. 2021 - May. 2023\\
& {\it GPA: 4.0}\\
& \\

& \textbf{Xidian University} \hfill Xian, Shaanxi\\
& B.E. in Telecommunications Engineering \hfill Sep. 2017 - Jun. 2021 \\
& {\it GPA: 3.6}\\
& \\

% --- Uncomment the next few lines if you want to include some courses ---
%& \textbf{Selected coursework}
%\begin{itemize}[noitemsep,leftmargin=*]
%\item \underline{Relevant subject 1}: Course 1, Course 2, Course 3, Course 4
%\item \underline{Relevant subject 2}: Course 1, Course 2, Course 3, Course 4
%\end{itemize} \\



% --- Section: Research experience ---

\nohyphens{\color{black}{Research experience}} 

&  \begin{itemize}[leftmargin=10pt, itemsep=-5pt, topsep=0pt,before=\textbf{North Carolina State University}]
    \item Mentors: Dongkuan Xu, Xipeng shen \hfill Aug. 2022 -- Present 
    \item Thesis: Zero-shot Code Generation via Rule-AI Co-learning from Document
    \item Contribution: Proposed a zero-shot code generation framework combining rule-based and AI-based methods to generate DSL code from document knowledge.
  \end{itemize}\\ 
  

% &  \begin{itemize}[leftmargin=10pt, itemsep=-5pt, topsep=0pt,before=\textbf{Temple University}]
%     \item Mentor: Wenpeng Yin\hfill May. 2022 - Jul. 2022
%     \item Thesis: Robustness of learning from task instructions\hfill
%     \item Contribution: Verified and analyzed the robustness of Tk-instruct on disturbed instructions.
%   \end{itemize}\\ 

&  \begin{itemize}[leftmargin=10pt, itemsep=-5pt, topsep=0pt,before=\textbf{University of Southern California}]
    \item Mentor: Massoud Pedram\hfill Aug. 2021 - Dec. 2021
    \item Thesis: Training Deep Neural Networks for Reduced-Memory-Access Inference\hfill
    \item Contribution: Integrated PyTorch distributed data-parallel framework into the flow to support multi-GPU processing.
  \end{itemize}\\ 
  
&  \begin{itemize}[leftmargin=10pt, itemsep=-5pt, topsep=0pt,before=\textbf{University of Southern California}]
    \item Mentor: Pedro Szekely\hfill Jan. 2022 - May. 2022
    \item Thesis: Integrating factual information from language models into knowledge graph embeddings \hfill 
    \item Contribution: Improved link prediction task by factual information mined from language models via prompts.
  \end{itemize}\\ 

&  \begin{itemize}[leftmargin=10pt, itemsep=-5pt, topsep=0pt,before=\textbf{ETH Zürich}]
    \item Mentor: Yuyi Wang\hfill June. 2020 - Oct. 2020
    \item Thesis: Construct pre-training data for text summarization based on trained metrics\hfill 
    \item Contribution: Use the trained metrics to replace ROUGE to construct the pre-training data needed for PEGASUS.
  \end{itemize}\\ 



% --- Section: Publications ---

\nohyphens{\color{black}{Publications}}
& \textbf{Few-shot Code Generation via Rule-AI Co-learning from Document} \\
& Jiasheng Gu, Zifan Nan, Dongkuan Xu, Xipeng shen \\
& \textit{In prep, ACL}\\
& \\


% & \textbf{Robustness of learning from task instructions} \\
% & Jiasheng Gu, Wenpeng Yin \\
% & \textit{Submitted, EACL}\\
% & \\

& \textbf{Artificial Intelligence Related Techniques Used in Recent Bio-medical Publications} \\
& Jiasheng Gu, Lili Wang, Soroush Vosoughi \\
& \textit{Submitted, JMIR}\\
& \\


% & \textit{Average student rating: X/5.} \\
& \\

% & \textbf{Teaching assistant, Department of Subject (University)} \hfill Spring 2020 \\
% & STAT 234: Name of course here \\
% & Topics and description of your responsibilities. Aliquam volutpat est vel massa. Sed dolor lacus, imperdiet non, ornare non, commodo eu, neque. \\
% & \textit{Average student rating: X/5.} \\
% & \\

% & \textbf{Teaching assistant, Department of Subject (University)} \hfill Spring 2020 \\
% & STAT 345: Name of course here \\
% & Topics and description of your responsibilities. Aliquam volutpat est vel massa. \\
% & \textit{Average student rating: X/5.} \\
& \\

% --- Section: Industry experience ---

{\color{black}{Industry experience}} 
& {\textbf{Lime}}\hfill Los Angeles, CA \\
& SDE internship \hfill Summer 2022 \\
& Reengineered a system for extracting and computing features, making it easier to modify feature definitions, compute features more efficiently, and add more tests.\\
&\\

& {\textbf{Transwarp}}\hfill Shanghai\\
& NLP internship \hfill Spring 2021 \\
& Established an NLP system to summarize the text through Tensorflow in the environment built by Nvidia Docker.\\
& \\

\newline
{\color{black}{Professional service}}
&  \begin{itemize}[leftmargin=10pt, itemsep=-5pt, topsep=0pt]
    \item ACM International Conference on Web Search and Data Mining \hfill 2023
  \end{itemize}\\ 

% --- Section: Teaching experience ---

{\color{black}{Teaching experience}} 
& \textbf{Teaching assistant, University of Southern California} \hfill Fall 2022 \\
& EE 503: Probability for Electrical and Computer Engineers \\
& Grading coursework and exams, leading and supervising lab exercises, and attending regular meetings. \\

% --- Section: Talks and tutorials ---

% {\color{black}{Talks and tutorials}} 
% & \textbf{Title of your most recent presentation} \hfill Month Year \\
% & Name of conference, workshop, seminar, venue, etc., or a description \\
% & \\

% & \textbf{Title of your second most recent presentation} \hfill Month Year \\
% & Name of conference, workshop, seminar, venue, etc., or a description \\
% & \\



% & \textbf{Languages} \\
% & Language 1 (fluent), Language 2 (advanced) \\
% & \\

% --- Section: Service and outreach ---

% \color{black}{Service and outreach}
% & \textbf{Title of organization you were in} \hfill Month Year -- Month Year \\
% & Description of your responsibilities. Integer pretium semper justo. Proin risus. Aliquam volutpat est vel massa. \\
% & \\

% --- Section: Professional society memberships ---
% --- Note: section title is spread over two lines ---

% {\color{black}{Professional}} 
% & {\textbf{Name of professional society.}} \hfill Month Year -- Present \\
% {\color{black}{memberships}} 
% & Some things you did or conferences you attended. Aliquam volutpat est vel massa. Sed dolor lacus, imperdiet non, ornare non, commodo eu, neque. \\
% & \\


% --- Section: Awards, scholarships, etc. ---
% --- Note: section title is spread over two lines ---

\newline
\nohyphens{{\color{black}{Honors and scholarships}}} 
&  \begin{itemize}[leftmargin=10pt, itemsep=-5pt, topsep=0pt]
    \item Masters Students Honors Program (University of Southern California) \hfill 2021
    \item Third Class Scholarship (Xidian University)\hfill 2019 
  \end{itemize}\\ 
% & Masters Students Honors Program (University of Southern California) \hfill 2020\\
% % {\color{black}{}} 
% & Third Class Scholarship (Xidian University)\hfill 2019 \\
% & \\


% --- Section: Various skills (programming, software, languages, etc.) ---

{\color{black}{Skills}} 
& \textbf{Programming}\\
& Python, C++, C, R, Java, SQL, JavaScript, HTML, MATLAB \\
& \textbf{Framework} \\
& PyTorch, Tensorflow, OpenCV, NumPy, Scikit-Learn, SciPy \\
& \textbf{Professional Softwares} \\
& Git, LaTeX, SPSS, Mathematica, AWS, GCP, Docker, MongoDB \\
& \\
% --- Section: Other interests/hobbies ---

% \nohyphens{\color{black}{Other interests}} & Some of your hobbies etc.\\

% --- End of CV! ---

\end{longtable}
\end{document}
