% --- LaTeX CV Template - S. Venkatraman ---

% Set document class and font size
\documentclass[letterpaper, 10pt]{article}
% \documentclass[11pt, letterpaper]{awesome-cv}
% \usepackage[utf8]{inputenc}

% Package imports
\usepackage{setspace, longtable, graphicx, hyphenat, hyperref, fancyhdr, ifthen, enumitem, amsmath, setspace}

% --- Page layout settings ---

% Set page margins
\usepackage[left=1in, right=1in, bottom=0.7in, top=0.7in]{geometry}

% Defines writer's homepage (optional)
% Usage: \homepage{<url>}
\newcommand*{\homepage}[1]{\def\@homepage{#1}}

% Defines writer's github (optional)
% Usage: \github{<github-nick>}
\newcommand*{\github}[1]{\def\@github{#1}}

% Defines writer's mobile (optional)
% Usage: \mobile{<mobile number>}
\newcommand*{\mobile}[1]{\def\@mobile{#1}}

% Defines writer's email (optional)
% Usage: \email{<email adress>}
\newcommand*{\email}[1]{\def\@email{#1}}





% Set line spacing
\renewcommand{\baselinestretch}{1.1}

% --- Page formatting ---

% Set link colors
\usepackage[dvipsnames]{xcolor}
\hypersetup{colorlinks=true, linkcolor=RoyalBlue, urlcolor=RoyalBlue}

% Set font to Libertine, including math support
\usepackage{libertine}
\usepackage[libertine]{newtxmath}

% Remove page numbering
\pagenumbering{gobble}

% --- Document starts here ---

\begin{document}

% Name and date of last update to this document
\noindent{\Huge{Jiasheng Gu}
\hfill{\it\footnotesize Updated \today}}

% --- Contact information and other items ---

\vspace{0.5cm} 
\begin{center}
\begin{tabular}{lllll}
% Line 1: Email, GitHub, office location
 \href{https://jiashenggu.github.io/}{\textbf{Homepage}}            &
 \href{https://github.com/jiashenggu}{\textbf{GitHub}}              &
 \href{https://www.linkedin.com/in/jiasheng-gu/}{\textbf{LinkedIn}} &
 \textbf{Email}: gujiashe@usc.edu      &
 \textbf{Phone}: +1 (213) 204-0294     

\end{tabular}
\end{center}

% --- Start the two-column table storing the main content ---

% Set spacing between columns
\setlength{\tabcolsep}{2pt}

% Set the width of each column
\begin{longtable}{p{1.3in}p{4.8in}}


% --- Section: Research Interests ---

\nohyphens{\color{black}{Research Interests}}
& My research interest lies in the fields of natural language processing and generation, along with reliable and efficient methods for interconverting between natural language and different forms of data. 

\textbf{Long-term research goal:} Use artificial intelligence to turn natural language into a bridge to various tasks. \\

\\

% --- Section: Education ---


% & \textbf{University 1} \hfill City, State \\ 
% & PhD in Subject \hfill Month Year -- Present \\
% & Mentors: Professors A, B. {\it GPA: X.YZ.}\\
% & \\

\color{black}{Education} & \textbf{University of Southern California} \hfill Los Angeles, CA \\
& M.S. in Machine Learning and Data Science \hfill Aug. 2021 - May. 2023\\
& Relevant Courses: Machine Learning, Deep Learning, Probability, Linear Algebra \\
& {\it GPA: 4.0/4.0 (100\%)}\\
& \\

& \textbf{Xidian University} \hfill Xian, Shaanxi\\
& B.E. in Telecommunications Engineering \hfill Sep. 2017 - Jun. 2021 \\
& {\it GPA: 3.6/4.0 (90\%, Top 10\%)}\\
& \\

% --- Uncomment the next few lines if you want to include some courses ---
%& \textbf{Selected coursework}
%\begin{itemize}[noitemsep,leftmargin=*]
%\item \underline{Relevant subject 1}: Course 1, Course 2, Course 3, Course 4
%\item \underline{Relevant subject 2}: Course 1, Course 2, Course 3, Course 4
%\end{itemize} \\

% --- Section: Publications ---

% \section*{\fontsize{9}{1}\selectfont Publications}
\nohyphens{\color{black}{Publications}}
& \textbf{Co-evolving Data-driven and NLU-driven Synthesizers for Generating Code in Domain Growth and Data Scarcity\label{few-shot_code}}  \\
& \textbf{Jiasheng Gu}, Zifan Nan, Zhiyuan Peng, Dongkuan Xu, Xipeng shen \\
& \textit{Submitted, ACL}\\
& \\

& \textbf{Robustness of learning from task instructions \label{robustness_instruction}} \\
& \textbf{Jiasheng Gu}, Hongyu Zhao, Hanzi Xu, Liangyu Nie, Hongyuan Mei, Wenpeng Yin \\
& \href{https://arxiv.org/abs/2212.03813}{\textit{Arxiv}}, Submitted, ACL\\
& \\

& \textbf{The Evolution of Artificial Intelligence in Bio-Medicine} \label{JMIR} \\
& \textbf{Jiasheng Gu}, Lili Wang, Soroush Vosoughi \\
& \textit{Submitted, Journal of Medical Internet Research (JMIR)}\\
& \\

% --- Section: Research experience ---

\nohyphens{\color{black}{Research Experience}} 
% \section*{\fontsize{9}{1}\selectfont Research Experience}

&  \begin{itemize}[leftmargin=10pt, itemsep=-5pt, topsep=0pt,before=\textbf{North Carolina State University}]
    \item Mentors: \href{http://personal.psu.edu/dux19/}{Dongkuan Xu}, \href{https://people.engr.ncsu.edu/xshen5/}{Xipeng Shen} \hfill Aug. 2022 -- Present 
    \item Project: Zero-shot Code Generation via Rule-AI Co-learning from Document
    \item Contribution: Proposed a zero-shot code generation framework combining rule-based and AI-based methods to generate DSL code.
    \item Publication: ACL in prep
  \end{itemize}\\ 

&  \begin{itemize}[leftmargin=10pt, itemsep=-5pt, topsep=0pt,before=\textbf{University of Southern California}]
    \item Mentors: \href{https://sites.usc.edu/eessc/people/}{Peter A. Beerel} \hfill Aug. 2022 -- Present 
    \item Project: Designing visual networks with very low FLOPs
    \item Contribution: Proposed a dilated depthwise convolution that captures global information and extensive experiments are done on it.
  \end{itemize}\\ 
  

&  \begin{itemize}[leftmargin=10pt, itemsep=-5pt, topsep=0pt,before=\textbf{Pennsylvania State University}]
    \item Mentor: \href{https://www.wenpengyin.org/}{Wenpeng Yin}\hfill June. 2022 - Oct. 2022
    \item Project: Robustness of learning from task instructions\hfill
    \item Contribution: Experimented and analyzed the robustness of the instruction-tuned model on perturbed instructions.
    \item Publication: \href{https://arxiv.org/abs/2212.03813}{\textit{Arxiv}}
  \end{itemize}\\ 

&  \begin{itemize}[leftmargin=10pt, itemsep=-5pt, topsep=0pt,before=\textbf{Dartmouth College}]
    \item Mentor: \href{https://www.cs.dartmouth.edu/~soroush//}{Soroush Vosoughi}\hfill May. 2022 - Sep. 2022
    \item Project: The Evolution of Artificial Intelligence in Bio-Medicine\hfill
    \item Contribution: Proposed a method to analyze artificial intelligence techniques used in biomedical publications.
    \item Publication: JMIR submitted
  \end{itemize}\\ 
  
&  \begin{itemize}[leftmargin=10pt, itemsep=-5pt, topsep=0pt,before=\textbf{University of Southern California}]
    \item Mentor: \href{https://usc-isi-i2.github.io/szekely/}{Pedro Szekely}\hfill Jan. 2022 - May. 2022
    \item Project: Integrating factual information from language models into knowledge graph embeddings \hfill 
    \item Contribution: Improved link prediction task by factual information mined from language models via prompts.
  \end{itemize}\\ 

&  \begin{itemize}[leftmargin=10pt, itemsep=-5pt, topsep=0pt,before=\textbf{University of Southern California}]
    \item Mentor: \href{https://mpedram.com/}{Massoud Pedram}\hfill Aug. 2021 - Dec. 2021
    \item Project: Reduced-Memory-Access Inference of Deep Neural Networks\hfill
    \item Contribution: Integrated PyTorch distributed data-parallel framework into the flow to support multi-GPU processing.
  \end{itemize}\\ 

&  \begin{itemize}[leftmargin=10pt, itemsep=-5pt, topsep=0pt,before=\textbf{ETH Zürich}]
    \item Mentor: \href{https://disco.ethz.ch/alumni/yuwang}{Yuyi Wang}\hfill June. 2020 - Oct. 2020
    \item Project: Designing pre-training tasks for text summarization\hfill 
    \item Contribution: Using trained metrics to find the highest importance sentences as summaries makes the pre-training task more effective.
  \end{itemize}\\ 






% & \textit{Average student rating: X/5.} \\


% & \textbf{Teaching assistant, Department of Subject (University)} \hfill Spring 2020 \\
% & STAT 234: Name of course here \\
% & Topics and description of your responsibilities. Aliquam volutpat est vel massa. Sed dolor lacus, imperdiet non, ornare non, commodo eu, neque. \\
% & \textit{Average student rating: X/5.} \\
% & \\

% & \textbf{Teaching assistant, Department of Subject (University)} \hfill Spring 2020 \\
% & STAT 345: Name of course here \\
% & Topics and description of your responsibilities. Aliquam volutpat est vel massa. \\
% & \textit{Average student rating: X/5.} \\


% --- Section: Industry experience ---

{\color{black}{Industry Experience}} 

&  \begin{itemize}[leftmargin=10pt, itemsep=-5pt, topsep=0pt,before=\textbf{Lime}]
    \item SDE internship \hfill May. 2022 - Aug. 2022 
    \item Reengineered a system for extracting and computing features to make it easier to modify feature definitions and compute features more efficiently.
  \end{itemize}\\ 

  
% & {\textbf{Lime}}\hfill Los Angeles, CA \\
% & SDE internship \hfill May. 2022 - Aug. 2022 \\
% & Reengineered a system for extracting and computing features, making it easier to modify feature definitions, compute features more efficiently, and add more tests.\\
% &\\

&  \begin{itemize}[leftmargin=10pt, itemsep=-5pt, topsep=0pt,before=\textbf{Umer Technology}]
    \item NLP internship \hfill Apr. 2021 - Aug. 2021 
    \item  Deployed a medical named entity identification system via BERT+CRF.
  \end{itemize}\\ 

&  \begin{itemize}[leftmargin=10pt, itemsep=-5pt, topsep=0pt,before=\textbf{Transwarp}]
    \item NLP internship \hfill Jan. 2021 - Apr. 2021 
    \item  Established a text summarization system with an extractive-abstractive structure.
  \end{itemize}\\ 


{\color{black}{Professional service}}
&  \begin{itemize}[leftmargin=10pt,topsep=0pt,before=\textbf{Program Committee}]
    \item The European Chapter of the ACL (EACL) \hfill 2023
    \item ACM International Conference on Web Search and Data Mining(WSDM) \hfill 2023
  \end{itemize}\\ 


% --- Section: Teaching experience ---

{\color{black}{Teaching Experience}} 
&  \begin{itemize}[leftmargin=10pt, itemsep=-5pt, topsep=0pt,before=\textbf{Teaching assistant, University of Southern California}]
    \item EE 503: Probability for Electrical and Computer Engineers \hfill Fall 2022
    \item  Grading course work and answering questions for students.
  \end{itemize}\\ 


% --- Section: Talks and tutorials ---

% {\color{black}{Talks and tutorials}} 
% & \textbf{Title of your most recent presentation} \hfill Month Year \\
% & Name of conference, workshop, seminar, venue, etc., or a description \\
% & \\

% & \textbf{Title of your second most recent presentation} \hfill Month Year \\
% & Name of conference, workshop, seminar, venue, etc., or a description \\
% & \\



% & \textbf{Languages} \\
% & Language 1 (fluent), Language 2 (advanced) \\
% & \\

% --- Section: Service and outreach ---

% \color{black}{Service and outreach}
% & \textbf{Title of organization you were in} \hfill Month Year -- Month Year \\
% & Description of your responsibilities. Integer pretium semper justo. Proin risus. Aliquam volutpat est vel massa. \\
% & \\

% --- Section: Professional society memberships ---
% --- Note: section title is spread over two lines ---

% {\color{black}{Professional}} 
% & {\textbf{Name of professional society.}} \hfill Month Year -- Present \\
% {\color{black}{memberships}} 
% & Some things you did or conferences you attended. Aliquam volutpat est vel massa. Sed dolor lacus, imperdiet non, ornare non, commodo eu, neque. \\
% & \\


% --- Section: Awards, scholarships, etc. ---
% --- Note: section title is spread over two lines ---


\nohyphens{{\color{black}{Awards}}} 
&  \begin{itemize}[leftmargin=10pt, itemsep=-5pt, topsep=0pt, before=\textbf{Master of Science (M.S.)}]
    \item Masters Students Honors Program (USC) \hfill 2021
    \end{itemize}\\ 
&  \begin{itemize}[leftmargin=10pt, itemsep=-5pt, topsep=0pt, before=\textbf{Bachelor of Engineering (B.E.)}]
    \item ZTE Algorithm Competition, Regional Winner Award \hfill 2020
    \item Third Class Scholarship (Xidian University)\hfill 2019 
  \end{itemize}\\ 
% & Masters Students Honors Program (University of Southern California) \hfill 2020\\
% % {\color{black}{}} 
% & Third Class Scholarship (Xidian University)\hfill 2019 \\
% & \\


% --- Section: Various skills (programming, software, languages, etc.) ---

{\color{black}{Skills}} 
& \textbf{Programming}\\
& Python, C++, C, R, Java, SQL, JavaScript, HTML, MATLAB \\
& \textbf{Framework} \\
& PyTorch, Tensorflow, OpenCV, NumPy, Scikit-Learn, SciPy \\
& \textbf{Professional Softwares} \\
& Git, LaTeX, SPSS, Mathematica, AWS, GCP, Docker, MongoDB \\
& \\

% --- Section: References ---

{\color{black}{References}} 
& \textbf{\href{https://usc-isi-i2.github.io/ilievski/}{Filip Ilievski}, Assistant Professor}\\
&  Viterbi School of Engineering \\
&  University of Southern California \\
&  ilievski@isi.edu \\
\\
& \textbf{\href{https://sites.usc.edu/eessc/people/}{Peter A. Beerel}, Professor}\\
&  Ming Hsieh Electrical and Computer Engineering Department \\
&  University of Southern California \\
&  pabeerel@usc.edu \\
% &  EEB 350, 3740 McClintock Ave., Los Angeles, CA 90089 \\
\\
& \textbf{\href{https://www.cs.dartmouth.edu/~soroush//}{Soroush Vosoughi}, Assistant Professor}\\
&  Department of Computer Science \\
&  Dartmouth College \\
&  soroush.vosoughi@dartmouth.edu \\
\\
& \textbf{\href{https://www.wenpengyin.org/}{Wenpeng Yin}, Assistant Professor}\\
&  Computer Science and Engineering Department \\
&  Pennsylvania State University \\
&  wenpeng.yin@temple.edu \\
% &  SERC376 1925 N 12th St, Philadelphia, PA 19122 \\
\\
& \textbf{\href{http://personal.psu.edu/dux19/}{Dongkuan Xu}, Assistant Professor}\\
&  Department of Computer Science \\
&  North Carolina State University \\
&  dxu27@ncsu.edu \\
% &  3258 EB II, 890 Oval Dr, Raleigh, NC 27695 \\
\\
& \textbf{\href{https://people.engr.ncsu.edu/xshen5/}{Xipeng Shen}, Professor}\\
&  Department of Computer Science \\
&  North Carolina State University \\
&  xshen5@ncsu.edu \\





% --- Section: Other interests/hobbies ---
% \nohyphens{\color{black}{Other interests}} & Some of your hobbies etc.\\

% --- End of CV! ---

\end{longtable}
\end{document}
