% --- LaTeX CV Template - S. Venkatraman ---

% Set document class and font size
\documentclass[letterpaper, 10pt]{article}
% \documentclass[11pt, letterpaper]{awesome-cv}
% \usepackage[utf8]{inputenc}

% Package imports
\usepackage{setspace, longtable, graphicx, hyphenat, hyperref, fancyhdr, ifthen, enumitem, amsmath, setspace}

% --- Page layout settings ---

% Set page margins
\usepackage[left=1in, right=1in, bottom=0.7in, top=0.7in]{geometry}

% Defines writer's homepage (optional)
% Usage: \homepage{<url>}
\newcommand*{\homepage}[1]{\def\@homepage{#1}}

% Defines writer's github (optional)
% Usage: \github{<github-nick>}
\newcommand*{\github}[1]{\def\@github{#1}}

% Defines writer's mobile (optional)
% Usage: \mobile{<mobile number>}
\newcommand*{\mobile}[1]{\def\@mobile{#1}}

% Defines writer's email (optional)
% Usage: \email{<email adress>}
\newcommand*{\email}[1]{\def\@email{#1}}


% Set line spacing
\renewcommand{\baselinestretch}{1.1}

% --- Page formatting ---

% Set link colors
\usepackage[dvipsnames]{xcolor}
\hypersetup{colorlinks=true, linkcolor=RoyalBlue, urlcolor=RoyalBlue}

% Set font to Libertine, including math support
\usepackage{libertine}
\usepackage[libertine]{newtxmath}

% Remove page numbering
\pagenumbering{gobble}

% --- Document starts here ---

\begin{document}

% Name and date of last update to this document
\noindent{\Huge{Jiasheng Gu}
\hfill{\it\footnotesize Updated \today}}

% --- Contact information and other items ---

\vspace{0.5cm} 
\begin{center}
\begin{tabular}{lllll}
% Line 1: Email, GitHub, office location
 \href{https://jiashenggu.github.io/}{\textbf{Homepage}}            &
 \href{https://github.com/jiashenggu}{\textbf{GitHub}}              &
 \href{https://www.linkedin.com/in/jiasheng-gu/}{\textbf{LinkedIn}} &
 \textbf{Email}: jiashengguwen@gmail.com      &
 \textbf{Phone}: +86 17701621592     

\end{tabular}
\end{center}

% --- Start the two-column table storing the main content ---

% Set spacing between columns
\setlength{\tabcolsep}{2pt}

% Set the width of each column
\begin{longtable}{p{1.3in}p{4.8in}}


% --- Section: Research Interests ---

\nohyphens{\color{black}{Research Interests}}
& My research interest lies in the fields of natural language processing and generation, along with reliable and efficient methods for interconverting between natural language and different forms of data. 

\textbf{Long-term research goal:} Use artificial intelligence to turn natural language into a bridge to various tasks. \\

\\

% --- Section: Education ---


\color{black}{Education} & \textbf{University of Southern California} \hfill Los Angeles, CA \\
& Honor M.S. in Machine Learning and Data Science \hfill Aug. 2021 - May. 2023\\
& Relevant Courses: Deep Learning, Probability, Linear Algebra, Parallel Computing \\
& {\it GPA: 4.0/4.0 (100\%)}\\
& \\

& \textbf{Xidian University} \hfill Xian, Shaanxi\\
& B.E. in Telecommunications Engineering \hfill Sep. 2017 - Jun. 2021 \\
& Relevant Courses: Data Structure, Calculus, Discrete Mathematics \\
& {\it GPA: 3.6/4.0 (90\%, Top 10\%)}\\
& \\

% --- Uncomment the next few lines if you want to include some courses ---
%& \textbf{Selected coursework}
%\begin{itemize}[noitemsep,leftmargin=*]
%\item \underline{Relevant subject 1}: Course 1, Course 2, Course 3, Course 4
%\item \underline{Relevant subject 2}: Course 1, Course 2, Course 3, Course 4
%\end{itemize} \\

% --- Section: Publications ---

% \section*{\fontsize{9}{1}\selectfont Publications}
\nohyphens{\color{black}{Publications}}\label{aclfinding2023}
& \textbf{Robustness of learning from task instructions \label{robustness_instruction}} \\
& \textbf{Jiasheng Gu}, Hongyu Zhao, Hanzi Xu, Liangyu Nie, Hongyuan Mei, Wenpeng Yin \\
& ACL 2023 findings, \href{https://arxiv.org/abs/2212.03813}{\textit{Arxiv}}\\
& \\

% & \textbf{The Evolution of Artificial Intelligence in Bio-Medicine} \label{JMIR} \\
% & \textbf{Jiasheng Gu}, Lili Wang, Soroush Vosoughi \\
% & \textit{Under review, Journal of Medical Internet Research (JMIR)}\\
% & \\
& \textbf{Co-evolving Data-driven and NLU-driven Synthesizers for Generating Code in Domain Growth and Data Scarcity\label{few-shot_code}}  \\
& \textbf{Jiasheng Gu}, Zifan Nan, Zhiyuan Peng, Dongkuan Xu, Xipeng Shen \\
& \textit{EMNLP workshop 2023}\\
& \\



% --- Section: Research experience ---

\nohyphens{\color{black}{Research Experience}} 
% \section*{\fontsize{9}{1}\selectfont Research Experience}

&  \begin{itemize}[leftmargin=10pt, itemsep=-5pt, topsep=0pt,before=\textbf{Pennsylvania State University}]
    \item Mentor: \href{https://www.wenpengyin.org/}{Wenpeng Yin}\hfill June. 2022 - Oct. 2022
    \item Project: Robustness of learning from task instructions\hfill
    \item Contribution: Experimented and analyzed the robustness of the instruction-tuned model on perturbed instructions.
    \item Publication: \href{https://arxiv.org/abs/2212.03813}{\textit{Arxiv}}
  \end{itemize}\\ 
  
&  \begin{itemize}[leftmargin=10pt, itemsep=-5pt, topsep=0pt,before=\textbf{North Carolina State University}]
    \item Mentors: \href{http://personal.psu.edu/dux19/}{Dongkuan Xu}, \href{https://people.engr.ncsu.edu/xshen5/}{Xipeng Shen} \hfill Aug. 2022 -- Present 
    \item Project: Zero-shot Code Generation via Rule-AI Co-learning from Document
    \item Contribution: Proposed a zero-shot code generation framework combining rule-based and AI-based methods to generate DSL code.
    \item Publication: EMNLP workshop submission
  \end{itemize}\\ 
  
&  \begin{itemize}[leftmargin=10pt, itemsep=-5pt, topsep=0pt,before=\textbf{Shanghai Jiaotong University}]
    \item Mentors: \href{http://pfliu.com/}{Pengfei Liu} \hfill April 2023 -- June 2023
    \item Project: Training Large Language Model on Math Tasks \href{https://github.com/GAIR-NLP/abel}{\textit{repo}}
    \item Contribution: Trained(continue pretraining and finetune) LLaMA-13B on well-designed math datasets to improve the performance of math tasks.
  \end{itemize}\\ 

&  \begin{itemize}[leftmargin=10pt, itemsep=-5pt, topsep=0pt,before=\textbf{University of Southern California}]
    \item Mentors: \href{https://sites.usc.edu/eessc/people/}{Peter A. Beerel} \hfill Aug. 2022 -- Dec. 2022
    \item Project: Designing visual networks with very low FLOPs
    \item Contribution: Proposed a dilated depthwise convolution that captures global information and extensive experiments are done on it.
  \end{itemize}\\ 
  
&  \begin{itemize}[leftmargin=10pt, itemsep=-5pt, topsep=0pt,before=\textbf{Dartmouth College}]
    \item Mentor: \href{https://www.cs.dartmouth.edu/~soroush//}{Soroush Vosoughi}\hfill May. 2022 - Sep. 2022
    \item Project: The Evolution of Artificial Intelligence in Bio-Medicine\hfill
    \item Contribution: Proposed a method to analyze artificial intelligence techniques used in biomedical publications.
    \item Publication: Under review, Journal of Medical Internet Research (JMIR)
  \end{itemize}\\ 
  

% &  \begin{itemize}[leftmargin=10pt, itemsep=-5pt, topsep=0pt,before=\textbf{University of Southern California}]
%     \item Mentor: \href{https://mpedram.com/}{Massoud Pedram}\hfill Aug. 2021 - Dec. 2021
%     \item Project: Reduced-Memory-Access Inference of Deep Neural Networks\hfill
%     \item Contribution: Integrated PyTorch distributed data-parallel framework into the flow to support multi-GPU processing.
%   \end{itemize}\\ 

&  \begin{itemize}[leftmargin=10pt, itemsep=-5pt, topsep=0pt,before=\textbf{ETH Zürich}]
    \item Mentor: \href{https://disco.ethz.ch/alumni/yuwang}{Yuyi Wang}\hfill June. 2020 - Oct. 2020
    \item Project: Designing pre-training tasks for text summarization\hfill 
    \item Contribution: Using trained metrics to find the highest-importance sentences as summaries makes the pre-training task more effective.
  \end{itemize}\\ 






% & \textit{Average student rating: X/5.} \\


% & \textbf{Teaching assistant, Department of Subject (University)} \hfill Spring 2020 \\
% & STAT 234: Name of course here \\
% & Topics and description of your responsibilities. Aliquam volutpat est vel massa. Sed dolor lacus, imperdiet non, ornare non, commodo eu, neque. \\
% & \textit{Average student rating: X/5.} \\
% & \\

% & \textbf{Teaching assistant, Department of Subject (University)} \hfill Spring 2020 \\
% & STAT 345: Name of course here \\
% & Topics and description of your responsibilities. Aliquam volutpat est vel massa. \\
% & \textit{Average student rating: X/5.} \\


% --- Section: Industry experience ---

{\color{black}{Industry Experience}} 

&  \begin{itemize}[leftmargin=10pt, itemsep=-5pt, topsep=0pt,before=\textbf{Lime}]
    \item SDE internship \hfill May. 2022 - Aug. 2022 
    \item Enhanced system scalability by refactoring the timed feature extraction and computation process. Achieved significant improvements in overall system efficiency through optimization of timed computations.
  \end{itemize}\\ 


&  \begin{itemize}[leftmargin=10pt, itemsep=-5pt, topsep=0pt,before=\textbf{Transwarp}]
    \item NLP internship \hfill Jan. 2021 - Apr. 2021 
    \item  Designed and implemented a hybrid long-text summarization system combining extractive and abstractive methods; utilized DGCNN for extractive summarization and leveraged BART for abstractive refinement.
  \end{itemize}\\ 


{\color{black}{Professional service}}
&  \begin{itemize}[leftmargin=10pt,topsep=0pt,before=\textbf{Program Committee}]
    \item The European Chapter of the ACL (EACL) \hfill 2023
    \item ACM International Conference on Web Search and Data Mining(WSDM) \hfill 2023
    \item Association for Computational Linguistics(ACL) \hfill 2023
    \item Conference on Empirical Methods in Natural Language Processing(EMNLP) \hfill 2023
    \item The Association for the Advancement of Artificial Intelligence(AAAI) \hfill 2023
  \end{itemize}\\ 


% --- Section: Teaching experience ---

{\color{black}{Teaching Experience}} 
&  \begin{itemize}[leftmargin=10pt, itemsep=-5pt, topsep=0pt,before=\textbf{Teaching assistant, University of Southern California}]
    \item EE 503: Probability for Electrical and Computer Engineers \hfill Fall 2022
  \end{itemize}\\ 


% --- Section: Talks and tutorials ---

% {\color{black}{Talks and tutorials}} 
% & \textbf{Title of your most recent presentation} \hfill Month Year \\
% & Name of conference, workshop, seminar, venue, etc., or a description \\
% & \\

% & \textbf{Title of your second most recent presentation} \hfill Month Year \\
% & Name of conference, workshop, seminar, venue, etc., or a description \\
% & \\



% & \textbf{Languages} \\
% & Language 1 (fluent), Language 2 (advanced) \\
% & \\

% --- Section: Service and outreach ---

% \color{black}{Service and outreach}
% & \textbf{Title of organization you were in} \hfill Month Year -- Month Year \\
% & Description of your responsibilities. Integer pretium semper justo. Proin risus. Aliquam volutpat est vel massa. \\
% & \\

% --- Section: Professional society memberships ---
% --- Note: section title is spread over two lines ---

% {\color{black}{Professional}} 
% & {\textbf{Name of professional society.}} \hfill Month Year -- Present \\
% {\color{black}{memberships}} 
% & Some things you did or conferences you attended. Aliquam volutpat est vel massa. Sed dolor lacus, imperdiet non, ornare non, commodo eu, neque. \\
% & \\


% --- Section: Awards, scholarships, etc. ---
% --- Note: section title is spread over two lines ---


\nohyphens{{\color{black}{Awards}}} 
&  \begin{itemize}[leftmargin=10pt, itemsep=-5pt, topsep=0pt, before=\textbf{USC Ming Hsieh Department of Electrical and Computer Engineering}]
    \item Masters Students Honors Program \hfill 2021
    % \item Third Class Scholarship \hfill 2019 
    \end{itemize}\\ 
% &  \begin{itemize}[leftmargin=10pt, itemsep=-5pt, topsep=0pt, before=\textbf{Bachelor of Engineering (B.E.)}]
%     \item ZTE Algorithm Competition, Regional Winner Award \hfill 2020
    
%   \end{itemize}\\ 
% & Masters Students Honors Program (University of Southern California) \hfill 2020\\
% % {\color{black}{}} 
% & Third Class Scholarship (Xidian University)\hfill 2019 \\
% & \\


% --- Section: Various skills (programming, software, languages, etc.) ---

{\color{black}{Skills}} 
& \textbf{Programming}\\
& Python, C++, C, R, Java, SQL, JavaScript, HTML, MATLAB \\
& \textbf{Framework} \\
& PyTorch, Tensorflow, OpenCV, NumPy, Scikit-Learn, SciPy \\
& \textbf{Professional Softwares} \\
& Git, LaTeX, SPSS, Mathematica, AWS, GCP, Docker, MongoDB \\
& \\


{\color{black}{References}} 
% --- Section: References ---
& \textbf{\href{https://www.wenpengyin.org/}{Wenpeng Yin}, Assistant Professor}\\
&  Computer Science and Engineering Department \\
&  Pennsylvania State University \\
&  wenpeng@psu.edu \\
% &  SERC376 1925 N 12th St, Philadelphia, PA 19122 \\
\\
& \textbf{\href{http://personal.psu.edu/dux19/}{Dongkuan Xu}, Assistant Professor}\\
&  Department of Computer Science \\
&  North Carolina State University \\
&  dxu27@ncsu.edu \\
% &  3258 EB II, 890 Oval Dr, Raleigh, NC 27695 \\
\\
& \textbf{\href{https://people.engr.ncsu.edu/xshen5/}{Xipeng Shen}, Professor}\\
&  Department of Computer Science \\
&  North Carolina State University \\
&  xshen5@ncsu.edu \\
\\
& \textbf{\href{https://sites.usc.edu/eessc/people/}{Peter A. Beerel}, Professor}\\
&  Ming Hsieh Electrical and Computer Engineering Department \\
&  University of Southern California \\
&  pabeerel@usc.edu \\
\\
% &  EEB 350, 3740 McClintock Ave., Los Angeles, CA 90089 \\
& \textbf{\href{https://www.cs.dartmouth.edu/~soroush//}{Soroush Vosoughi}, Assistant Professor}\\
&  Department of Computer Science \\
&  Dartmouth College \\
&  soroush.vosoughi@dartmouth.edu \\



% --- Section: Other interests/hobbies ---
% \nohyphens{\color{black}{Other interests}} & Some of your hobbies etc.\\

% --- End of CV! ---

\end{longtable}
\end{document}
